\documentclass[xcolor={dvipsnames,table},c,compress,colorlinks]{beamer}
\usepackage{xspace}
\usepackage{comment}
\usepackage{tikz}
\usetikzlibrary{shapes,arrows,positioning}
\usepackage{colortbl}
\graphicspath{{art-images}}

%----------------------------------------------------------------------
% Set up the listings package
\usepackage{listings,bera}
\lstset{ basicstyle=\ttfamily\color{white}% for 'normal' code
       , xleftmargin=0.5em%         left indent, so line number isn't lost
       , xrightmargin=0.5em%        right indent
       , numbers=none%              numbering is not needed on short slides
       , tabsize=2%                 don't waste space with huge tabs
       , stringstyle=\color{yellow}%            for delimeted strings
       , showstringspaces=false%                do not use undercap as space 
       , keywordstyle=\color{green}%            for identified keywords
       , commentstyle=\itshape\color{SkyBlue}%  for comments
       }
\lstloadlanguages{[ISO]C++, bash, Perl}
\lstdefinelanguage{FHiCL}
{
  sensitive = true,
  morekeywords={T, F, true, false, BEGIN_PROLOG, END_PROLOG},
  morecomment=[l]{\#},
  morestring=[b]",
  % moredelim=[s]{\{}{\}},
  % moredelim=[s]{[}{]},
  otherkeywords={:,@local::,\#include}
}
\lstnewenvironment{cpp}{\lstset {language=[ISO]C++}}{}
\lstnewenvironment{fhicllang}{\lstset {language=FHiCL}}{} 
%----------------------------------------------------------------------
% Font selection
\usepackage[T1]{fontenc}
\usepackage{textcomp}
\usepackage{palatino}

%----------------------------------------------------------------------
% Beamer theme control
%----------------------------------------------------------------------
\setbeamercolor*{structure}{fg=OliveGreen!70!black}
\useinnertheme{rounded}
\usecolortheme{dark}
\usefonttheme{structureitalicserif}
\usefonttheme{serif}
%  Turn off the navigation symbols
\setbeamertemplate{navigation symbols}{}
\setbeamertemplate{footline}[frame number]
% Use greyed-out overlays
\setbeamercovered{invisible}

%----------------------------------------------------------------------
% My own macros.
%----------------------------------------------------------------------
%\newcommand{\productname}[1]{\textsc{#1}\xspace}
\newcommand{\productname}[1]{\textbf{\textcolor{cyan}{#1}}\xspace}
\newcommand{\cmd}[1]{\texttt{#1}\xspace}
\newcommand{\expt}[1]{\textbf{\textcolor{Goldenrod}{#1}}\xspace}
\newcommand{\art}{\productname{art}}
\newcommand{\artdaq}{\productname{artdaq}}
\newcommand{\Rprod}{\productname{R}}
\newcommand{\Sprod}{\productname{S}}
\newcommand{\blankline}{\vskip\baselineskip}
\newcommand{\coverpage}[1]{\begin{frame}\begin{center}\vfill\Large #1\end{center}\end{frame}}
\newcommand{\IF}{\expt{Intensity Frontier}}
\newcommand{\Q}[1]{\textcolor{green}{#1}\xspace}
\newcommand{\A}[1]{#1\xspace}
\newcommand{\cppcode}[1]{\lstinline[language=[ISO]C++,escapechar=|]{#1}}
\newcommand{\term}[1]{\texttt{\bfseries\color{green} #1}\xspace}
\newcommand{\eg}{\textit{e.g.}}
\newcommand{\etc}{\textit{etc.}}
\newcommand{\nova}{\expt{NO$\nu$A}}
\newcommand{\argoneut}{\expt{ArgoNeuT}}
\newcommand{\muboone}{\expt{$\mu$BooNE}}
\newcommand{\larsoft}{\expt{LArSoft}}
\newcommand{\mg}{\expt{Muon g-2}}
\newcommand{\mue}{\expt{Mu2e}}
\newcommand{\cms}{\expt{CMS}}
\newcommand{\rootprod}{\productname{ROOT}}
\newcommand{\cling}{\productname{Cling}}
\newcommand{\dzero}{\expt{D\O}}
\newcommand{\mpi}{\productname{MPI}}
\newcommand{\cmake}{\productname{CMake}}
\newcommand{\make}{\productname{GNU Make}}
\newcommand{\python}{\productname{Python}}
\newcommand{\sqlite}{\productname{SQLite}}
\newcommand{\dsf}{\expt{DS50}}
\newcommand{\lbne}{\expt{LBNE}}
\newcommand{\fhicl}{\productname{FHiCL}}
\newcommand{\ruby}{\productname{Ruby}}
\newcommand{\ups}{\productname{UPS}}
\newcommand{\rups}{\productname{Relocatable UPS}}
\newcommand{\srt}{\productname{SoftRelTools}}
\newenvironment{trythis}%
  {\begin{block}{Try this now!}}%
  {\end{block}}

\newenvironment{sourcefile}%
  {\begin{block}{C++ code}}%
  {\end{block}}
%----------------------------------------------------------------------
% Structural elements for the talk.
%
%    TODO: Learn how to control the institution template, so that
%          it doesn't show up too small.
%----------------------------------------------------------------------

\title{Development tools for \art.}
\author[Green]{Chris Green}
\institute[Fermilab]{SCD-ADSS-SSI}
\date[]{\today}

\usepackage{pgf}
\logo{\pgfputat{\pgfxy(1.2,7.)}{\pgfbox[center,base]{\includegraphics[width=2.in]{fermi-logo-white}}}}
%----------------------------------------------------------------------
% Document begins here...
%----------------------------------------------------------------------

\begin{document}
% Title page
\begin{frame}[fragile,plain]
  \color{structure}
  \begin{columns}[onlytextwidth,b]
    \begin{column}{0.7\textwidth}
      \parbox[t][][c]{\textwidth}{\hspace{.5in}\includegraphics[width=0.4\textwidth]{flyingmachine_l}}
      \\ \vspace{.3in}
      \Huge\textit{Development tools for \art}
    \end{column}
    \begin{column}{0.3\textwidth}
      \hfill
      \mbox{\includegraphics{monalisa_full}}
    \end{column}
  \end{columns}
  \vspace{.3in}
  % \begin{tabular}[b]{lr}
  \begin{columns}[onlytextwidth,b]
    % \parbox[b]{.42\linewidth}{

    \begin{column}{0.45\textwidth}
      Chris Green \\
      SCD-ADSS-SSI
    \end{column}
      % }
    % & %
    \begin{column}{0.55\textwidth}
      \hfill\mbox{\includegraphics[width=2.in]{fermi-credit-white}} \\
    \end{column}
  \end{columns}
  % \end{tabular}
\end{frame}

% Outline
\begin{frame}\frametitle{Outline}
  \begin{itemize}
  \item Motivation.
  \item Feature overview.
  \item Details.
  \item Developer experience.
  \item Ongoing developments.
  \end{itemize}
\end{frame}

\begin{frame}\frametitle{Why more build tools?}
  \begin{itemize}
  \item Package / install. \\
    \IF experiments want the switchability of UPS but without the
    \cmd{ups declare} commands, and the ability to install multiple
    packages from tarball.
  \item Build.
    \begin{itemize}
    \item Parallel builds.
    \item Support for testing: scripts, execs, packages, dependent
      tests, pass / fail criteria.
    \item Build consistency (very important).
    \item Ease of setup / use.
    \item Must be able to provide packages usable by other build system (vital).
    \end{itemize}
  \end{itemize}
\end{frame}

\begin{frame}\frametitle{Feature overview}
  \begin{itemize}
  \item Use existing products as basis: \ups, \cmake,
    \productname{CTest}, \make.
  \item Hierarchical package system with rigid version and qualifier
    dependency checking to ensure binary-compatible packages and
    consistent applications (compiler version, debug, language
    standard).
  \item Easy install of all or part of linked packages via tarball.
  \item Simple developer setup, tailored directives to specify build
    products, dependencies, tests, \etc.
  \item Seamless parallel build / test.
  \item Non-viral: \art suite is \ups packages, use of \cmake not
    required by experiments.
  \item All builds out-of-source: same source supports co-existing debug
    / profile builds.
  \end{itemize}
\end{frame}

\begin{frame}\frametitle{Details: \rups}
  \begin{itemize}
  \item Same \ups, new capabilities.
  \item No \cmd{prd/}, \cmd{db/} directories
  \item Each product version has a \cmd{\$\{PROD\_VER\}.version}
    directory instead.
  \item Use product dependencies (with \cmd{-B} option, \cmd{+}
    qualifiers) to ensure compatible products with error-on-failure.
  \end{itemize}
\end{frame}

\begin{frame}\frametitle{Details: \cmake}
  \begin{itemize}
  \item Built-in support for parallel builds, test.
  \item Flexible, modular, higher-level than \make.
  \item Excellent dependency management.
  \item Install, package, package configuration facilities mesh well
    with \rups.
  \end{itemize}
\end{frame}

\begin{frame}\frametitle{Details: \productname{cetbuildtools}}
  \begin{itemize}
  \item \cmake-based.
  \item Simple product configuration with \cmd{product\_deps} file,
    listing external package dependencies.
  \item Straightforward \cmake macros to do most common things:
    libraries, execs, \art plugins, \rootprod dictionaries, tests with
    pass / fail criteria \etc.
  \item Each package built separately and installed as a \rups product
    to guarantee a consistent build.
  \item Auxiliary scripts for things like version bumps, coherent
    release builds, \etc.
  \end{itemize}
\end{frame}

\begin{frame}\frametitle{Developer experience}
  \begin{itemize}
  \item Used to build, package and deliver the \art suite to \IF
    experiments.
  \item Being used on a small-scale by \mg.
  \item Full \art suite build / test / install / package with $>200$
    tests in 8:20 (vs 48:55 serial). A NOP build of \art only followed by
    tests is 0:34 (4:04 serial). 
  \item Able to specify unit tests up through integration tests very
    easily, including arguments, dependent test chains, required files,
    \etc.
  \end{itemize}
\end{frame}

\begin{frame}\frametitle{Ongoing development}
  \begin{itemize}
  \item Automatic generation of a package's installed library list for
    use by other packages.
  \item A safe scheme for reliable simultaneous multi-package
    builds: test releases \textit{a la} \srt are notorious for allowing
    inconsistent builds. This in turn may necessitate:
  \item Library versioning / checking scheme integral to the build
    system.
  \item Sell to \IF experiments.
  \end{itemize}
\end{frame}

\end{document}

%%% Local Variables: 
%%% mode: latex
%%% TeX-master: t
%%% End: 
